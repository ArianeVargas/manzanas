2.    Descripción del contexto de la necesidad
 
Las Manzanas del Cuidado son espacios de la ciudad en los que se brinda tiempo y servicios a las mujeres y a sus familias.
En las Manzanas del Cuidado las cuidadoras tienen tiempo y servicios gratuitos, aquí pueden estudiar, emprender, emplearse, descansar, ejercitarse, recibir orientación y asesoría jurídica y psicológica, lavar su ropa y la de su familia en lavanderías comunitarias, todo totalmente gratis.
 
Las Manzanas son áreas de la ciudad en las que se tiene la infraestructura y servicios para atender de manera próxima y simultánea a las cuidadoras y a sus familias. Por ejemplo, en una Manzana del Cuidado las cuidadoras y quienes ellas cuidan pueden encontrar colegios, jardines, parques, hospitales, centros de atención para personas mayores.
 
3.    Requerimientos
 
Se debe realizar un sistema de información web que tenga en cuenta los siguientes requisitos:
·	El sistema debe permitir el registro de Municipios que cuentan con una o varias manzanas del cuidado. 
·	Las manzanas deben registrar: Código, nombre, localidad y dirección
·	El sistema debe permitir ver las manzanas en un mapa y mostrar una tarjeta con la información de estas.
·	Las manzanas cuentas con diferentes tipos de servicios, categorías de servicios y servicios de los cuales se requiere mínimo información como código, nombre, descripción
·	Así mismo de cada servicio puede contar con establecimientos que les prestan la infraestructura y dar un mejor servicio con más calidad. De los establecimientos se desea guardar un código, nombre, responsable, dirección.
·	Una vez se haya realizado el registro en el sistema se debe contar con una página que permita a la mujer cuidadora proponer la manzana, servicio y día y hora que podría asistir.
·	De las mujeres se requiere guardar: Tipo de documento, Documento, Nombres, Apellidos, teléfono, correo electrónico, Ciudad, dirección, ocupación, servicios en los que le gustaría participar 
·	El sistema debe tener un Login, un método para restaurar contraseña 
·	Todos los formularios deben contar con validaciones y mensajes de alerta
·	El sistema de información debe permitir hacer el CRUD para cada uno de los módulos
·	La asignación de la manzana se debe dar por cercanía y/o servicio que desee de la ubicación de la mujer cuidadora.
·	Se deben generar reportes de cada módulo y tener la opción de imprimir en Excel o pdf
 
 
4.    Diseño
 
Para el diseño o interfaz el aprendiz puede hacer la búsqueda de todo el material multimedia que desee y considere que es acorde para cada uno de los módulos, donde proporcione facilidad al usuario para su interacción.
El diseño de la interfaz debe centrarse en proporcionar una experiencia de usuario intuitiva y atractiva. Se pueden utilizar elementos multimedia relevantes para cada módulo, asegurándose de que la navegación sea sencilla y amigable. El diseño debe enfocarse en la usabilidad y la accesibilidad para que las mujeres cuidadoras puedan interactuar de manera efectiva con el sistema.
Además, se debe considerar la incorporación de elementos visuales que reflejen la naturaleza de apoyo y cuidado de las Manzanas, transmitiendo confianza y comodidad a las usuarias.


**Pitch: Solución Innovadora para el Cuidado de las Mujeres - Manzana del Cuidado**

*Introducción:*
¿Te imaginas un lugar donde las mujeres cuidadoras y sus familias puedan recibir apoyo y servicios de manera gratuita, cerca de su hogar? Esa es la visión de Manzana del Cuidado, y nuestro software es la solución innovadora que hace que esta visión sea una realidad.

*La Necesidad:*
En un mundo donde las mujeres cuidadoras a menudo enfrentan desafíos abrumadores, desde el cuidado de sus seres queridos hasta la búsqueda de tiempo para sí mismas, Manzana del Cuidado es la respuesta. Nuestra plataforma aborda la necesidad crítica de brindar apoyo a estas mujeres y sus familias, ofreciendo servicios esenciales como orientación legal y psicológica, lavandería comunitaria, y acceso a instalaciones cercanas como colegios y hospitales. La necesidad de un espacio como Manzana del Cuidado es innegable.

*La Solución:*
Lo que hace que nuestro software sea excepcional es su capacidad para unir comunidades y brindar un servicio integral. Con Manzana del Cuidado, puedes registrar municipios y sus manzanas, visualizar fácilmente la ubicación de estas en un mapa y acceder a información detallada en tarjetas informativas. Además, gestionar diferentes tipos de servicios y establecimientos que los respaldan es sencillo. Pero lo más importante es la posibilidad de que las mujeres cuidadoras propongan su asistencia y reciban asignaciones basadas en su ubicación y preferencias de servicio.

*Nuestra Estrategia de Venta:*
La clave de nuestro éxito es la empatía y el compromiso con nuestro propósito. Nos conectamos con organizaciones gubernamentales, grupos de apoyo y comunidades locales para difundir la palabra y ganar su respaldo. Ofrecemos planes de suscripción flexibles y accesibles para garantizar que todas las mujeres cuidadoras puedan acceder a nuestros servicios. Además, brindamos soporte y capacitación integral para garantizar una adopción fluida.

*Funcionalidades Destacadas:*
- Registro de municipios y manzanas con información detallada.
- Visualización de manzanas en un mapa interactivo.
- Gestión de servicios y establecimientos.
- Propuestas de asistencia por parte de las mujeres cuidadoras.
- Asignaciones basadas en proximidad y preferencias.
- Generación de informes y exportación a Excel o PDF.
- Interfaz intuitiva y amigable para una experiencia sin complicaciones.

En resumen, Manzana del Cuidado y nuestro software están revolucionando la forma en que apoyamos a las mujeres cuidadoras y sus familias. Estamos aquí para hacer que la carga sea más ligera y empoderar a estas mujeres para que puedan alcanzar sus metas personales y profesionales. Únete a nosotros en este viaje para construir un mundo más cuidadoso y equitativo.